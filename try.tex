\documentclass[a4, 12pt]{article}

\usepackage[utf8]{inputenc}
\usepackage[slovene]{babel}
\usepackage{graphicx}
\usepackage{amsmath}
\usepackage{amsfonts}
\usepackage{latexsym}
\usepackage{amssymb}
\usepackage{listings}

\begin{document}

\title{ Osem vzporedni kvadrati  \\
  \large Kratek opis projekta}

\author{Avtorja: Sebastjan Šenk, Ana Marija Okorn}

\maketitle

\section{Navodilo}

Glede na niz osno vzporednih kvadratov v ravnini poiščite točko v ravnini, ki se dotika čim več kvadratov. Za rešitev te težave uporabite CLP. Izvedite poskuse, da ustvarite kvadrate, ki so vzporedni z osjo, in poiščete optimalne rešitve. Kaj pa iskanje črte, ki se dotika čim več kvadratov? 

\section{Opis problema}

Podano imamo množico P, ki vsebuje n kvadratov, katerih stranice morajo biti vzporedne koordinatinm osem. S pomočjo celoštevilskega linearnega programiranja (CLP) morava posikati točko oziroma točke (če jih je več), ki bo v preseku največ kvadratov. kasneje pa bova poiskušala poiskati še premico, ki se bo dotikala največ kvadratov.






\end{document}



