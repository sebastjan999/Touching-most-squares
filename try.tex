\documentclass[a4, 12pt]{article}

\usepackage[utf8]{inputenc}
\usepackage[slovene]{babel}
\usepackage{graphicx}
\usepackage{amsmath}
\usepackage{amsfonts}
\usepackage{latexsym}
\usepackage{amssymb}
\usepackage{listings}

\begin{document}

\title{ Osem vzporedni kvadrati  \\
  \large Kratek opis projekta}

\author{Avtorja: Sebastjan Šenk, Ana Marija Okorn}

\maketitle

\section{Navodilo}

Glede na niz osno vzporednih kvadratov v ravnini poiščite točko v ravnini, ki se dotika čim več kvadratov. Za rešitev te težave uporabite CLP. Izvedite poskuse, da ustvarite kvadrate, ki so vzporedni z osjo, in poiščete optimalne rešitve. Kaj pa iskanje črte, ki se dotika čim več kvadratov? 

\section{Opis problema}



Podano imamo množico, ki vsebuje enotske kvadrate, katerih stranice morajo biti vzporedne koordinatinm osem in njihova dolžina enaka 1.
ž Te kvadrate bova generirala iz naključno izbranih točk na kordinatni osi, kjer bo\[ x \in [0,10]\]  in \[y \in [0,10].\] 
Kvadrate bova generirala s pomočjo dolžine stranice. Najprej bova naredila eksperiment z manjšim številom kvadratom, kasneje pa
bova pogledala tudi za večje množice kvadratov. S pomočjo celoštevilskega linearnega programiranja (CLP) morava poiskati točko 
oziroma točke (če jih je več), ki se bo dotikala največ kvadratov. Torej to kar bova iskala bo presek največ kvadratov. Kasneje 
pa bova poiskušala poiskati še premico, ki se bo dotikala največ kvadratov. Za to težavo bova za vsako premico y = kx + n, za vsak 
k s pomočjo programa preštela koliko kvadratov premica seka. Pri tem si bova pomagala tudi z logičnimi izrazi, ki jih bova zapisala 
kot linearne enačbe.





\end{document}



